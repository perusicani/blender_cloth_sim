% !TeX encoding = windows-1250


{ \large 
\vspace{15pt}
% prilagodite ovu izjavu s obzirom na potrebni rod imenice (izradio ili izradila)
%::::::::::::::::::::::::::::::::::::::::::::::::::::::::::::::::::::::::::::::::
Izjavljujem da sam samostalno izradila ovaj rad. \vspace{7cm} \\ 

%::::::::::::::::::::::::::::::::::::::::::::::::::::::::::::::::::::::::::::::::
% ovo ispod nema potrebe dirati!
%:::::::::::::::::::::::::::::::::::::::::::::::::::::::::::::::::::::::::::::::::::
\noindent Rijeka, \MONTH~\thisyear.   
\hspace{5.5cm}
	\verb|_______________|  % pomoću duzine ove crte regulirati potrebnu sirinu crte za potpis

\begin{flushright}
	\vspace{-15pt}
	Ani Perušić
	\verb|      |   % pomocu praznih mjesta unutar | | crta, regulirati poravnjanje duzine imena i prezimena s gornjom crtom za potpis
\end{flushright}
%:::::::::::::::::::::::::::::::::::::::::::::::::::::::::::::::::::::::::::::::::::

} % \large